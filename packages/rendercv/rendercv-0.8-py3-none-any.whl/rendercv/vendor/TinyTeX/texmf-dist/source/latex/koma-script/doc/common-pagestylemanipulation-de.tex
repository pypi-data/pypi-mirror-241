% ======================================================================
% common-pagestylemanipulation-de.tex
% Copyright (c) Markus Kohm, 2013-2022
%
% This file is part of the LaTeX2e KOMA-Script bundle.
%
% This work may be distributed and/or modified under the conditions of
% the LaTeX Project Public License, version 1.3c of the license.
% The latest version of this license is in
%   http://www.latex-project.org/lppl.txt
% and version 1.3c or later is part of all distributions of LaTeX 
% version 2005/12/01 or later and of this work.
%
% This work has the LPPL maintenance status "author-maintained".
%
% The Current Maintainer and author of this work is Markus Kohm.
%
% This work consists of all files listed in MANIFEST.md.
% ======================================================================
%
% Text that is common for several chapters of the KOMA-Script guide
% Maintained by Markus Kohm
%
% ============================================================================

\KOMAProvidesFile{common-pagestylemanipulation-de.tex}
                 [$Date: 2023-06-16 10:32:27 +0200 (Fr, 16. Jun 2023) $
                  KOMA-Script guide (common paragraph: 
                                     Setting up defined page styles)]

\section{Beeinflussung von Seitenstilen}
\seclabel{pagestyle.content}
\BeginIndexGroup
\BeginIndex{}{Seiten>Stil}

\IfThisCommonLabelBase{scrlayer}{%
  Obwohl \Package{scrlayer} selbst keine konkreten Seitenstile mit Inhalt
  definiert -- die erwähnten Seitenstile
  \DescRef{\LabelBase.pagestyle.@everystyle@} und \PageStyle{empty} werden ja
  zunächst ohne Ebenen, also leer definiert --, stellt es einige Optionen und
  Befehle zur Beeinflussung von Inhalten zur Verfügung.%
}{%
  \IfThisCommonLabelBase{scrlayer-scrpage}{%
    In \autoref{sec:scrlayer-scrpage.predefined.pagestyles} wurde erklärt, wie
    die Seitenstile \DescRef{\LabelBase.pagestyle.scrheadings} und
    \DescRef{\LabelBase.pagestyle.plain.scrheadings} grundlegend vordefiniert
    sind und wie diese Vorbelegung grundsätzlich geändert werden kann. Es
    fehlen jedoch noch Informationen, wie beispielsweise die Kolumnentitel
    zustande kommen, wie man die Breite des Kopfes und Fußes verändern kann
    und wie man Linien über oder unter Kopf oder Fuß setzen kann. Obwohl dies
    eigentlich Fähigkeiten des Pakets
    \hyperref[cha:scrlayer]{\Package{scrlayer}}%
    \important{\hyperref[cha:scrlayer]{\Package{scrlayer}}} sind, werden sie
    nachfolgend erläutert, da diese grundlegenden Eigenschaften von
    \hyperref[cha:scrlayer]{\Package{scrlayer}} einen wichtigen Teil der
    Möglichkeiten von \IfThisCommonLabelBase{scrlayer-scrpage}{%
      \Package{scrlayer-scrpage}%
    }{%
      \hyperref[cha:scrlayer-scrpage]{\Package{scrlayer-scrpage}}%
    } ausmachen.%
  }{%
    \IfThisCommonLabelBase{scrlayer-scrpage-experts}{%
      Dieser Abschnitt ist als Ergänzung zu
      \autoref{sec:scrlayer.pagestyle.content} zu verstehen und beschreibt
      Dinge, die sich dem Anfänger nicht unbedingt sofort erschließen. Wenn
      Sie jenen Abschnitt bereits gelesen und verstanden haben, können Sie
      natürlich auch direkt mit
      \autoref{sec:\LabelBase.pagestyle.content.next} auf
      \autopageref{sec:\LabelBase.pagestyle.content.next} forfahren.%
    }{\InternalCommonFileUsageError}%
  }%
}

\IfThisCommonLabelBase{scrlayer-scrpage-experts}{\iffalse}{%
  \csname iftrue\endcsname}%
  \begin{Declaration}
    \IfThisCommonLabelBase{scrlayer}{%
      \Option{automark}
      \OptionVName{autooneside}{Ein-Aus-Wert}
      \Option{manualmark}
    }{}%
    \Macro{automark}\OParameter{Gliederungsebene der rechten Marke}
                    \Parameter{\mbox{Gliederungsebene der linken Marke}}
    \Macro{automark*}\OParameter{Gliederungsebene der rechten Marke}
                     \Parameter{Gliederungsebene der linken Marke}
    \Macro{manualmark}
  \end{Declaration}
  \IfThisCommonLabelBase{scrlayer-scrpage}{%
    \begin{Explain}
      Bei den Standardklassen und auch bei den \KOMAScript-Klassen fällt die
      Entscheidung, ob mit lebenden oder statischen
      Kolumnentiteln\Index{Kolumnentitel>lebend}\Index{Kolumnentitel>statisch}
      gearbeitet werden soll, über die Wahl des entsprechenden
      Seitenstils. Wie bereits in \autoref{sec:maincls.pagestyle} erklärt,
      erhält man bei Wahl von
      \DescRef{maincls.pagestyle.headings}\IndexPagestyle{headings}%
      \important{\DescRef{maincls.pagestyle.headings}} lebende
      Kolumnentitel. Unter lebenden Kolumnentiteln versteht man die
      Wiederholung eines für die Seite oder die \emph{Kolumne} markanten
      Textes meist im Kopf, seltener im Fuß der Seite.

      Bei den Artikel-Klassen\OnlyAt{\Class{article}\and \Class{scrartcl}}
      \Class{article} oder \hyperref[cha:maincls]{\Class{scrartcl}} wird für
      den lebenden Kolumnentitel\textnote{lebende Kolumnentitel} im
      einseitigen Modus die Abschnittsüberschrift, also das obligatorische
      oder das optionale Argument von \DescRef{maincls.cmd.section}
      verwendet. Diese wird als \emph{rechte Marke} behandelt. Im
      doppelseitigen Satz wird dieselbe Überschrift als \emph{linke Marke}
      verwendet und gleichzeitig die Unterabschnittsüberschrift als
      \emph{rechte Marke}. Ausgegeben wird die linke Marke, wie der Name schon
      sagt, auf der linken Seite, während die rechte Marke auf rechten Seiten
      -- im einseitigen Modus also auf allen Seiten -- ausgegeben wird. Beim
      Setzen der linken Marke für den Abschnitt werden von den Klassen in der
      Voreinstellung außerdem auch immer die rechten Marken gelöscht.

      Bei den Bericht- und Buch-Klassen\OnlyAt{\Class{report}\and
        \Class{scrreprt}\and \Class{book}\and \Class{scrbook}} wird eine Ebene
      höher begonnen. Im einseitigen Modus wird also die Kapitelüberschrift
      als rechte Marke gesetzt. Im doppelseitigen Satz wird die
      Kapitelüberschrift als linke Marke und die Abschnittsüberschrift als
      rechte Marke gesetzt.

      Verwendet man hingegen als Seitenstil
      \DescRef{maincls.pagestyle.myheadings}\IndexPagestyle{myheadings}%
      \important{\DescRef{maincls.pagestyle.myheadings}}%
      \textnote{manuelle Kolumnentitel}, so existieren zwar die Marken im Kopf
      genauso und auch die Seitenzahlen werden gleich platziert, allerdings
      werden die Marken nicht automatisch durch die Überschriften gesetzt. Man
      kann sie dann nur manuell über die später in diesem Abschnitt
      dokumentierten Anweisungen\important{%
        \DescRef{\ThisCommonLabelBase.cmd.markright}\\
        \DescRef{\ThisCommonLabelBase.cmd.markboth}}
      \DescRef{\ThisCommonLabelBase.cmd.markright} und
      \DescRef{\ThisCommonLabelBase.cmd.markboth} befüllen.
    \end{Explain}\par%
    Genau diese Unterscheidung wurde bei %
    \iffalse \Package{scrpage2} und nun auch bei \fi%
    \hyperref[cha:scrlayer]{\Package{scrlayer}}\IndexPackage{scrlayer}%
    \IfThisCommonLabelBase{scrlayer-scrpage}{%
      \important{\hyperref[cha:scrlayer]{\Package{scrlayer}}}}{}
    aufgehoben. Statt die Unterscheidung zwischen
    automatischen\textnote{lebende und manuelle Kolumnentitel} und manuellen
    Kolumnentiteln über den Seitenstil vorzunehmen, gibt es die beiden
    Anweisungen \Macro{automark} und \Macro{manualmark}.%
  }{%
    \IfThisCommonLabelBase{scrlayer}{%
      Bei den meisten Klassen bestimmt die Wahl des Seitenstils, meist
      \PageStyle{headings} und \PageStyle{myheadings}, darüber, ob die
      Kolumnentitel automatisch oder manuell erzeugt werden. Bei
      \Package{scrlayer} erfolgt die Unterscheidung stattdessen über die
      beiden Anweisungen \Macro{automark} und \Macro{manualmark}.%
    }{\InternalCommonFileUsageError}%
  }%

  Mit \Macro{manualmark}\important{\Macro{manualmark}} wird dabei auf manuelle
  Marken umgeschaltet. Es deaktiviert also das automatische Setzen der
  Marken. Demgegenüber kann mit \Macro{automark}\important{\Macro{automark}}
  und \Macro{automark*} festgelegt werden, welche Gliederungsebenen für das
  automatische Setzen der Marken verwendet werden sollen. Das optionale
  Argument gibt dabei die \PName{Gliederungsebene der rechten Marke} an,
  während das obligatorische Argument die \PName{Gliederungsebene der linken
    Marke} ist. Als Argument werden jeweils die Namen der Gliederungsebenen
  angegeben, also \PValue{part}, \PValue{chapter}, \PValue{section},
  \PValue{subsection}, \PValue{subsubsection}, \PValue{paragraph} oder
  \PValue{subparagraph}.

  Normalerweise sollte die höhere Ebene die linke Marke setzen, während die
  tiefere Ebene für die rechte Marke zu verwenden ist. Diese übliche
  Konvention ist jedoch keine Pflicht, sondern lediglich sinnvoll.

  Es ist zu beachten\textnote{Achtung!}, dass nicht alle Klassen Kolumnentitel
  für alle Ebenen ermöglichen. So setzen die
  Standardklassen\textnote{\KOMAScript{} vs. Standardklassen} beispielsweise
  nie Kolumnentitel für \DescRef{maincls.cmd.part}. Die \KOMAScript-Klassen
  unterstützen hingegen alle Ebenen.

  Der Unterschied zwischen \Macro{automark} und
  \Macro{automark*}\important{\Macro{automark*}} liegt darin, dass
  \Macro{automark} alle vorherigen Befehle zum automatischen Setzen der Marken
  aufhebt, während die Stern-Version \Macro{automark*} lediglich die Aktionen
  für die angegebenen Gliederungsebenen ändert.%
  \IfThisCommonLabelBase{scrlayer-scrpage}{}{ %
    \iffalse% Umbruchoptimierung
    Man kann so, wie in \autoref{sec:scrlayer-scrpage.pagestyle.content} ab
    \PageRefxmpl{scrlayer-scrpage.mark} gezeigt, auch relativ komplexe Fälle
    abdecken. %
    %\else%
    Beispiele für die Verwendung finden Sie in
    \autoref{sec:scrlayer-scrpage.pagestyle.content} ab
    \PageRefxmpl{scrlayer-scrpage.mark}. %
    \fi%
  }% Umbruchoptimierung
  \IfThisCommonLabelBase{scrlayer-scrpage}{\iftrue}{\csname
    iffalse\endcsname}%
    \iffalse% Umbruchkorrekturtext
      \iffree{}{\par
        Das veraltete Paket
        \Package{scrpage2}\IndexPackage{scrpage2}\important{\Package{scrpage2}}
        kennt sowohl \Macro{manualmark} als auch \Macro{automark}, jedoch nicht
        \Macro{automark*}. Daher sind die nachfolgenden Beispielen nicht
        vollständig auf die Verwendung von \Package{scrpage2} übertragbar.%
      }%
    \fi
    % 
    \begin{Example}
      \phantomsection\xmpllabel{mark}%
      Angenommen Sie wollen, dass wie üblich auf den linken Seiten eines
      Buches die Kapitelüberschriften als automatische Kolumnentitel verwendet
      werden und auf den rechten Seiten die
      Abschnittsüberschriften. Allerdings soll auf rechte Seiten so lange
      ebenfalls die Kapitelüberschrift verwendet werden, bis der erste
      Abschnitt auf"|taucht. Dazu wird zuerst
      \IfThisCommonLabelBase{scrlayer-scrpage}{%
        \Package{scrlayer-scrpage}%
      }{%
        \hyperref[cha:scrlayer-scrpage]{\Package{scrlayer-scrpage}}%
      } geladen. Dadurch ist auch bereits automatisch der Seitenstil
      \DescRef{\LabelBase.pagestyle.scrheadings} aktiviert. Das Dokument
      beginnt also mit:
\begin{lstcode}
  \documentclass{scrbook}
  \usepackage{scrlayer-scrpage}
\end{lstcode}
      Als nächstes wird dafür gesorgt, dass die Kapitelüberschriften sowohl
      die linke als auch die rechte Marke setzen:
\begin{lstcode}
  \automark[chapter]{chapter}
\end{lstcode}
      Dann sollen die Abschnittsüberschriften zusätzlich die rechten Marken
      setzen:
\begin{lstcode}
   \automark*[section]{}
\end{lstcode}
      Hier findet die Stern-Version Anwendung, da die vorherige
      \Macro{automark}-Anweisung weiterhin wirksam bleiben soll. Außerdem
      bleibt das Argument für die \PName{Gliederungsebene der linken Marke}
      leer, weil diese Marke unverändert bleiben soll.

      Alles, was jetzt noch fehlt, ist ein wenig Dokumentinhalt:
\begin{lstcode}
  \usepackage{lipsum}
  \begin{document}
  \chapter{Kapitel}
  \lipsum[1-20]
  \section{Abschnitt}
  \lipsum[21-40]
  \end{document}
\end{lstcode}
      \IfThisCommonLabelBase{scrlayer-scrpage}{}{% Umbruchkorrektur
      Dabei ist das Paket \Package{lipsum}\IndexPackage{lipsum} mit seiner
      Anweisung \Macro{lipsum}\IndexCmd{lipsum} sehr nützlich.}%

      Wenn Sie dieses Beispiel einmal testen, werden Sie sehen, dass die
      Kapitelanfangsseite wie üblich ohne Kolumnentitel ist, da sie
      automatisch im \PageStyle{plain}-Seitenstil
      \DescRef{\LabelBase.pagestyle.plain.scrheadings} gesetzt wird (siehe
      dazu Anweisung \DescRef{maincls.cmd.chapterpagestyle} auf
      \DescPageRef{maincls.cmd.chapterpagestyle}). Seite~2 bis 4 tragen als
      Kolumnentitel die Kapitelüberschrift. Nachdem auf Seite~4 eine
      Abschnittsüberschrift ausgegeben wurde, ändert sich der Kolumnentitel
      auf Seite~5 in die Abschnittsüberschrift. Ab dann werden die beiden
      Überschriften im Kopf wechselweise ausgegeben, auf linken Seiten die
      Kapitelüberschrift, auf rechten Seiten die Abschnittsüberschrift.%
    \end{Example}
  \fi

  \IfThisCommonLabelBase{scrlayer}{}{%
  \begin{Declaration}
    \Option{automark}
    \OptionVName{autooneside}{Ein-Aus-Wert}
    \Option{manualmark}
  \end{Declaration}}
  Außer mit den \IfThisCommonLabelBase{scrlayer}{}{zuvor erklärten }Befehlen
  kann auch direkt mit den beiden Optionen
  \Option{manualmark}\important{\Option{manualmark}\\\Option{automark}} und
  \Option{automark} zwischen automatischen und manuellen Kolumnentiteln hin-
  und hergeschaltet werden. Dabei verwendet \Option{automark} bei Klassen mit
  \DescRef{maincls.cmd.chapter}-Anweisung immer die
  Voreinstellung\textnote{Voreinstellung}
  \IfThisCommonLabelBase{scrlayer-scrpage}{\iftrue}{\csname
    iffalse\endcsname}%
    \lstinline|\automark[section]{chapter}| und bei anderen Klassen
    \lstinline|\automark[subsection]{section}|.
  \else
\begin{lstcode}
  \automark[section]{chapter}
\end{lstcode}
  und bei anderen Klassen:
\begin{lstcode}
  \automark[subsection]{section}
\end{lstcode}
  \fi

  \IfThisCommonLabelBaseOneOf{scrlayer,scrlayer-scrpage}{% Umbruchkorrektur
    Im einseitigen Modus\textnote{einseitiger Satz} will man in der Regel,
    dass nur die höheren Ebenen den Kolumnentitel vorgeben.%
  }{%
    Im einseitigen Modus\textnote{einseitiger Satz} will man in der Regel
    nicht, dass die untergeordnete Ebene die rechte Marke
    beeinflusst. Stattdessen soll auch mit der Voreinstellung nur die höhere
    Ebene, die beispielsweise im doppelseitigen Modus in der Voreinstellung
    alleine die linke Marke beeinflusst, den Kolumnentitel aller Seiten
    vorgeben.%
  } Diese Voreinstellung entspricht einer aktiven Option
  \Option{autooneside}\important{\Option{autooneside}}. Die Option versteht
  die Werte für einfache Schalter, die in \autoref{tab:truefalseswitch} auf
  \autopageref{tab:truefalseswitch} angegeben sind. Wird die Option
  deaktiviert, so wirken sich im einseitigen Satz sowohl das optionale als
  auch das obligatorische Argument auf den Kolumnentitel aus.%
  \IfThisCommonLabelBase{scrlayer-scrpage}{\iftrue}{\csname
    iffalse\endcsname}%
    \begin{Example}
      \phantomsection\xmpllabel{mark.oneside}%
      Angenommen, Sie wollen im einseitigen Modus eines Berichts %
      \iffalse % Umbruchkorrektur
      eine ganz ähnliche Verwendung des Kolumnentitels erreichen wie im
      vorherigen Beispiel. Konkret soll so lange die Kapitelüberschrift
      verwendet werden, %
      \else %
      so lange die Kapitelüberschriften für den Kolumnentitel verwenden, %
      \fi %
      bis ein Abschnitt gesetzt wird. Ab dann soll nur noch die
      Abschnittsüberschrift verwendet werden. %
      \iftrue % Umbruchkorrektur
      Dazu wird das %
      \iftrue vorherige \fi % Umbruchkorrektur
      Beispiel wie folgt abgewandelt:%
      \fi %
\begin{lstcode}
  \documentclass{scrreprt}
  \usepackage[autooneside=false]{scrlayer-scrpage}
  \automark[section]{chapter}
  \usepackage{lipsum}
  \begin{document}
  \chapter{Kapitel}
  \lipsum[1-20]
  \section{Abschnitt}
  \lipsum[21-40]
  \end{document}
\end{lstcode}
      \iffalse % Umbruchkorrektur
      Wie zu sehen ist, wird in diesem Fall keine %
      \else %
      In diesem Fall wird keine %
      \fi
      \iffalse ergänzende \fi % Umbruchkorrektur
      \DescRef{\LabelBase.cmd.automark*}-Anweisung benötigt. Sie sollten zum
      Vergleich die Option \Option{autooneside} %
      \iffalse auch \fi % Umbruchkorrektur
      einmal %
      \iffalse auf \PValue{true} setzen oder sie \fi % Umbruchkorrektur
      entfernen. Ein Unterschied ist dann ab Seite~4 %
      \iffalse im Kolumnentitel im Kopf der Seiten \fi % Umbruchkorrektur
      zu sehen.
    \end{Example}
  \fi

  Das\textnote{Achtung!} Laden des Pakets selbst hat übrigens noch keine
  Auswirkung darauf, ob mit automatischen Kolumnentiteln gearbeitet wird oder
  nicht. Erst die explizite Verwendung einer der Optionen \Option{automark}
  oder \Option{manualmark} oder einer der beiden Anweisungen
  \DescRef{\LabelBase.cmd.automark} oder \DescRef{\LabelBase.cmd.manualmark}
  schafft hier klare Verhältnisse.%
  \IfThisCommonLabelBase{scrlayer}{\par%
    \iffalse% Umbruchkorrektur
    Bei Bedarf finden Sie weitere Beispiele zur Verwendung dieser Befehle und
    Optionen mit dem auf \Package{scrlayer} basierenden Paket
    \hyperref[cha:scrlayer-scrpage]{\Package{scrlayer-scrpage}}
    \else
    Beispiele finden Sie
    \fi
    in \autoref{sec:scrlayer-scrpage.pagestyle.content}, ab
    \DescPageRef{scrlayer-scrpage.cmd.manualmark}.%
  }{%
    \EndIndexGroup%
  }%    
  \EndIndexGroup


  \IfThisCommonLabelBase{scrlayer}{% Bereits bei \layercontentsmeasure erklärt.
  }{%
    \begin{Declaration}
      \OptionVName{draft}{Ein-Aus-Wert}
    \end{Declaration}
    Die \KOMAScript-Option versteht die Standardwerte für einfache Schalter,
    die in \autoref{tab:truefalseswitch} auf \autopageref{tab:truefalseswitch}
    angegeben sind. Ist die Option aktiviert, so werden
    \IfThisCommonLabelBase{scrlayer}{für die Entwurfsphase}{\unskip} alle
    Elemente der Seitenstile zusätzlich mit
    Maßlinien\Index{Masslinien=Maßlinien} versehen.%
    \IfThisCommonLabelBaseOneOf{scrlayer,scrlayer-scrpage}{}{ % Umbruchkorrektur
      Dies kann während der Entwurfsphase manchmal nützlich sein.}%
    \IfThisCommonLabelBase{scrlayer-scrpage}{ %
      Falls diese Option global gesetzt wurde, die Maßlinien aber nicht
      gewünscht sind, kann die Option auch nur für das Paket deaktiviert
      werden, indem man \OptionValue{draft}{false} %
      \iffalse % Umbruchkorrektur
      als optionales Argument von \DescRef{\LabelBase.cmd.usepackage} %
      \fi %
      beim Laden des Pakets angibt.%
    }{}%
    \EndIndexGroup%
  }%


  \begin{Declaration}
    \Macro{MakeMarkcase}\Parameter{Text}
    \OptionVName{markcase}{Wert}
  \end{Declaration}
  Die automatischen, nicht jedoch die manuellen Kolumnentitel verwenden
  \Macro{MakeMarkcase} für ihre Ausgabe. Ist die Anweisung beim Laden von
  \IfThisCommonLabelBase{scrlayer}{%
    \Package{scrlayer}%
  }{%
    \hyperref[cha:scrlayer]{\Package{scrlayer}}%
  } nicht\textnote{bedingte Voreinstellung} definiert, so
  \IfThisCommonLabelBase{scrlayer}{gibt sie in der Voreinstellung ihr Argument
    \PName{Text} unverändert aus}{wird sie in der Voreinstellung derart
    definiert, dass sie ihr Argument \PName{Text} unverändert
    ausgibt}. \IfThisCommonLabelBase{scrlayer}{Die}{Diese} Voreinstellung kann
  jedoch durch Umdefinierung von \Macro{MakeMarkcase} geändert
  werden. \IfThisCommonLabelBase{scrlayer}{Dies}{Die Umdefinierung} kann auch
  automatisch durch Verwendung von Option \Option{markcase} mit einem der
  Werte aus \autoref{tab:scrlayer-scrpage.markcase}%
  \IfThisCommonLabelBase{scrlayer-scrpage}{}{,
    \autopageref{tab:scrlayer-scrpage.markcase}} erfolgen.%
  \IfThisCommonLabelBase{scrlayer-scrpage}{%
    \begin{table}
      \centering
      \caption[Mögliche Werte für Option \Option{markcase}]{Mögliche Werte für
        Option \Option{markcase} zur Wahl von Groß-/Kleinschreibung in
        automatischen Kolumnentiteln}%
      \label{tab:\ThisCommonLabelBase.markcase}%
      \begin{desctabular}
        \pventry{lower}{\IndexOption[indexmain]{markcase~=lower}%
          definiert \DescRef{\LabelBase.cmd.MakeMarkcase} so um, dass
          automatische Kolumnentitel mit Hilfe von \Macro{MakeLowercase} in
          Kleinbuchstaben gewandelt werden (Minuskelsatz).%
        }\\[-1.7ex]
        \pventry{upper}{\IndexOption[indexmain]{markcase~=upper}%
          definiert \DescRef{\LabelBase.cmd.MakeMarkcase} so um, dass
          automatische Kolumnentitel mit Hilfe von \Macro{MakeUppercase} in
          Großbuchstaben gewandelt werden (Versalsatz).%
        }\\[-1.7ex]
        \pventry{title}{\IndexOption[indexmain]{markcase~=title}%
          \IfThisCommonLabelBase{scrlayer}{%
            \ChangedAt{v3.41}{\Package{scrlayer}}}{%
            \IfThisCommonLabelBase{scrlayer-scrpage}{%
              \ChangedAt{v3.41}{\Package{scrlayer-scrpage}}}}{}%
          definiert \DescRef{\LabelBase.cmd.MakeMarkcase} so um, dass
          automatische Kolumnentitel mit Hilfe von \Macro{MakeTitlecase}
          gewandelt werden (\emph{Englischer Titelsatz}). Dabei werden Nummer
          und Titel getrennt behandelt. Ist \Macro{MakeTitlecase} wegen eines
          zu alten \LaTeX{} Kerns nicht definiert, wird eine Warnung
          ausgegeben und die Option ignoriert.%
        }\\[-1.7ex]
        \pventry{used}{\IndexOption[indexmain]{markcase~=used}%
          definiert \DescRef{\LabelBase.cmd.MakeMarkcase} so um, dass für
          automatische Kolumnentitel keine automatische Veränderung der
          Groß-/Kleinschreibung durchgeführt wird.%
        }\\[-1.7ex]
        \entry{\PValue{ignoreuppercase}, \PValue{nouppercase},
          \PValue{ignoreupper},
          \PValue{noupper}}{\IndexOption[indexmain]{markcase~=noupper}%
          definiert nicht nur \DescRef{\LabelBase.cmd.MakeMarkcase} so um,
          dass für automatische Kolumnentitel keine automatische Veränderung
          der Groß-/Kleinschreibung durchgeführt wird, sondern deaktiviert
          zusätzlich lokal für alle Ebenen aller Seitenstile
          \Macro{MakeUppercase} und \Macro{uppercase}.%
        }%
      \end{desctabular}
    \end{table}
  }{}%

  \IfThisCommonLabelBase{scrlayer}{%
    Aufgrund der mangelnden typografischen Qualität der primitiven Umwandlung
    in Großbuchstaben (siehe die Erklärung zu
    \DescRef{scrlayer-scrpage.option.markcase} in
    \autoref{sec:scrlayer-scrpage.pagestyle.content} auf
    \autopageref{expl:scrlayer-scrpage.MakeUppercase}) empfiehlt der
    \KOMAScript-Autor den Verzicht auf Versalsatz.%
  }{%
    Leider\phantomsection\label{expl:\ThisCommonLabelBase.MakeUppercase}
    führt die von \LaTeX{} für Versalsatz\Index{Versalsatz} vorgesehene
    Anweisung \Macro{MakeUppercase}\IndexCmd{MakeUppercase} zu keinem guten
    Ergebnis, da weder gesperrt noch ausgeglichen wird. Dies liegt teilweise
    sicher daran, dass für typografisch korrekten Versalsatz eine
    Glyphenanalyse notwendig ist, um die konkrete Form der Buchstaben\iffree{
      und ihrer Kombinationen}{} in den Ausgleich der Sperrung einfließen zu
    lassen. Der \KOMAScript-Autor empfiehlt daher, auf Versalsatz für die
    Kolumnentitel zu verzichten.%
  } Dies ist normalerweise mit
  \OptionValue{markcase}{used}\important{\OptionValue{markcase}{used}}%
  \IndexOption[indexmain]{markcase~=used} möglich. Allerdings fügen einige
  Klassen selbst beispielsweise bei den Kolumnentitel für Verzeichnisse ein
  \Macro{MakeUppercase} oder sogar die \TeX-Anweisung \Macro{uppercase}
  ein. Für diese Fälle gibt es auch noch die Einstellung
  \OptionValue{markcase}{noupper}\important{\OptionValue{markcase}{noupper}}%
  \IndexOption[indexmain]{markcase~=noupper}, mit deren Hilfe
  \Macro{MakeUppercase} und \Macro{uppercase} für die Kolumnentitel lokal
  deaktiviert werden können.
  \EndIndexGroup
\fi
  

\IfThisCommonLabelBase{scrlayer-scrpage}{\iffalse}{\csname iftrue\endcsname}
  \iffree{\begin{Declaration}}{\begin{Declaration}[0]}%
    \Macro{righttopmark}
    \Macro{rightbotmark}
    \Macro{rightfirstmark}
    \Macro{rightmark}
    \Macro{lefttopmark}
    \Macro{leftbotmark}
    \Macro{leftfirstmark}
    \Macro{leftmark}
  \end{Declaration}
  \LaTeX\ChangedAt{v3.16}{\Package{scrlayer}} verwendet für die Seitenstile
  normalerweise eine zweiteilige \TeX-Marke. Im Kolumnentitel kann auf den
  linken Teil der Marke mit \Macro{leftmark}\important{\Macro{leftmark}}
  zugegriffen werden, während der rechte Teil der Marke über
  \Macro{rightmark}\important{\Macro{rightmark}} verfügbar ist. Tatsächlich
  ist es wohl auch so gedacht, dass \Macro{leftmark} für linke Seiten und
  \Macro{rightmark} für rechte Seiten im doppelseitigen Druck verwendet
  wird\IfThisCommonLabelBase{scrlayer-scrpage-experts}{, während im
    einseitigen Layout nur rechte Marken gesetzt werden}{. Im einseitigen
    Layout setzen die Gliederungsbefehle der Standardklassen den linken Teil
    der Marke hingegen gar nicht erst.}% Umbruchkorrektur

  \TeX{} selbst kennt drei Möglichkeiten, auf eine Marke zuzugreifen.
  \Macro{botmark}\IndexCmd{botmark}\important{\Macro{botmark}} ist die auf der
  zuletzt zusammengestellten Seite zuletzt gültige Marke. Das entspricht der
  letzten gesetzten Marke der Seite. Wurde auf der Seite keine Marke gesetzt,
  so entspricht es der zuletzt gesetzten Marke auf den bereits ausgegebenen
  Seiten. Die \LaTeX-Anweisung \Macro{leftmark}\important{\Macro{leftmark}}
  verwendet genau diese Marke, gibt also den linken Teil der letzten Marke der
  Seite aus. Dies entspricht ebenfalls
  \Macro{leftbotmark}\important{\Macro{leftbotmark}}. Im Vergleich dazu gibt
  \Macro{rightbotmark}\important{\Macro{rightbotmark}} den rechten Teil dieser
  Marke aus.

  \Macro{firstmark}\IndexCmd{firstmark}\important{\Macro{firstmark}} ist die
  erste Marke der zuletzt zusammengestellten Seite. Das entspricht der ersten
  Marke, die auf der Seite gesetzt wurde. Wurde auf der Seite keine Marke
  gesetzt, so entspricht es der zuletzt gesetzten Marke auf den bereits
  ausgegebenen Seiten. Die \LaTeX-Anweisung
  \Macro{rightmark}\important{\Macro{rightmark}} verwendet genau diese Marke,
  gibt also den rechten Teil der ersten Marke der Seite aus. Dies entspricht
  ebenfalls \Macro{rightfirstmark}\important{\Macro{rightfirstmark}}. Im
  Vergleich dazu gibt \Macro{leftfirstmark}\important{\Macro{leftfirstmark}}
  den linken Teil dieser Marke aus.

  \Macro{topmark}\IndexCmd{topmark}\important{\Macro{topmark}} ist der Inhalt,
  den \Macro{botmark} hatte, bevor die aktuelle Seite zusammengestellt
  wurde. \LaTeX{} verwendet dies selbst nie.
  \IfThisCommonLabelBase{scrlayer}{%
    \Package{scrlayer}%
  }{%
    \hyperref[cha:scrlayer]{\Package{scrlayer}}%
  } bietet die Möglichkeit, mit
  \Macro{lefttopmark}\important{\Macro{lefttopmark}} den linken Teil
  dieser Marke und mit \Macro{righttopmark}\important{\Macro{righttopmark}}
  den rechten Teil auszugeben.

  Es\textnote{Achtung!} ist zu beachten, dass der linke und rechte Teil der
  Marke immer nur gemeinsam gesetzt werden kann. Selbst wenn man mit
  \DescRef{scrlayer.cmd.markright}\IndexCmd{markright} nur den rechten Teil
  verändert, wird der linke Teil (unverändert) mitgesetzt. Entsprechend
  setzen im doppelseitigen Layout die höheren Gliederungsebenen beim
  Seitenstil
  \PageStyle{headings}\important{\PageStyle{headings}}\IndexPagestyle{headings}
  immer beide Teile. Beispielsweise verwendet
  \DescRef{scrlayer.cmd.chaptermark} dann
  \DescRef{scrlayer.cmd.markboth} mit einem leeren rechten Argument. Das ist
  auch der Grund, warum \Macro{rightmark} beziehungsweise
  \Macro{rightfirstmark} auf der Seite einer Kapitelüberschrift immer einen
  leeren Wert zurückgibt, selbst wenn danach beispielsweise über
  \DescRef{scrlayer.cmd.sectionmark} oder indirekt über
  \DescRef{maincls.cmd.section} ein neuer rechter Teil
  \IfThisCommonLabelBase{scrlayer-scrpage}{der Marke}{\unskip} gesetzt
  wurde.

  Bitte\textnote{Achtung!} beachten Sie, dass die Verwendung einer der hier
  erklärten Anweisungen zur Ausgabe des linken oder rechten Teils der Marke
  innerhalb einer Seite zu unerwarteten Ergebnissen führen kann. Sie sind
  wirklich nur zur Verwendung im Kopf oder Fuß eines Seitenstils
  gedacht. Daher sollten sie bei \IfThisCommonLabelBase{scrlayer}{%
    \Package{scrlayer}%
  }{%
    \hyperref[cha:scrlayer]{\Package{scrlayer}}%
  } immer Teil des Inhalts
  einer Ebene sein. Dagegen spielt es keine Rolle, ob sie auf den Hintergrund
  oder den Vordergrund beschränkt werden, da alle Ebenen erst nach der
  Zusammenstellung der aktuellen Seite ausgegeben werden.

  Näheres zum Mark-Mechanismus \iffree{von \TeX{}}{\unskip} ist beispielsweise
  \cite[Kapitel~23]{knuth:texbook} zu entnehmen. Das Thema ist dort als
  absolutes Expertenwissen markiert.\IfThisCommonLabelBase{scrlayeralsonot}{
    % Umbruchkorrektur
    Sollte Sie obige Erklärung also eher
  verwirrt haben, machen Sie sich bitte nichts daraus.}{}%
  \EndIndexGroup
\fi
  

\IfThisCommonLabelBase{scrlayer-scrpage-experts}{\iffalse}{%
  \csname iftrue\endcsname}%
  \begin{Declaration}
    \IfThisCommonLabelBase{scrlayer-scrpage}{%
      \Macro{leftmark}
      \Macro{rightmark}
    }{}%
    \Macro{headmark}
    \Macro{pagemark}
  \end{Declaration}
  \IfThisCommonLabelBase{scrlayer-scrpage}{%
    Will man von den vordefinierten Seitenstilen abweichen, so muss man in der
    Regel auch selbst entscheiden können, wo die Marken gesetzt werden
    sollen. Mit \Macro{leftmark}\important{\Macro{leftmark}} platziert man die
    linke Marke. Diese wird dann bei der Ausgabe der Seite durch den
    entsprechenden Inhalt ersetzt.

    Dementsprechend kann man mit
    \Macro{rightmark}\important{\Macro{rightmark}} die rechte Marke
    platzieren, die dann bei der Ausgabe der Seite durch den entsprechenden
    Inhalt ersetzt wird. Für einige Feinheiten dabei sei auch auf die
    weiterführenden Erklärungen zu \DescRef{maincls-experts.cmd.rightmark} in
    \autoref{sec:maincls-experts.addInfos},
    \DescPageRef{maincls-experts.cmd.rightmark} verwiesen.

  }{}%

  Mit \Macro{headmark}\important{\Macro{headmark}} kann man sich das Leben
  erleichtern. Diese Erweiterung von \IfThisCommonLabelBase{scrlayer}{%
    \Package{scrlayer}%
  }{%
    \hyperref[cha:scrlayer]{\Package{scrlayer}}%
  } entspricht je nachdem, ob die aktuelle Seite eine linke oder rechte ist,
  \IfThisCommonLabelBase{scrlayer-scrpage}{\Macro{leftmark} oder
    \Macro{rightmark}}{\DescRef{\LabelBase.cmd.leftmark} oder
    \DescRef{\LabelBase.cmd.rightmark}}.

  Die Anweisung \Macro{pagemark}\important{\Macro{pagemark}} hat genau
  genommen nichts mit den Marken von \TeX{} zu tun. Sie dient dazu, eine
  formatierte Seitenzahl zu platzieren.
  \BeginIndex{FontElement}{pagenumber}\LabelFontElement{pagenumber}%
  Bei ihrer Ausgabe wird dann auch die Schrifteinstellung für das Element
  \FontElement{pagenumber}\important{\FontElement{pagenumber}}
  verwendet. Diese kann mit Hilfe der Anweisungen
  \DescRef{maincls.cmd.setkomafont} und \DescRef{maincls.cmd.addtokomafont}
  verändert werden (siehe auch \autoref{sec:maincls.textmarkup},
  \DescPageRef{maincls.cmd.setkomafont}).%
  \EndIndex{FontElement}{pagenumber}%
  \IfThisCommonLabelBase{scrlayer-scrpage}{\iftrue}{%
    \par%
    Für ein Beispiel \iffalse zur Verwendung von
    \Macro{headmark} und \Macro{pagemark} \fi % Umbruchkorrektur
    siehe
    \autoref{sec:scrlayer-scrpage.pagestyle.content},
    \PageRefxmpl{scrlayer-scrpage.headmark}.%
    \csname iffalse\endcsname%
  }%
    \begin{Example}
      \phantomsection\xmpllabel{headmark}%
      Angenommen, Sie wollen, dass auch im einseitigen Modus der Kolumnentitel
      immer am linken Rand und die Seitenzahl immer am rechten Rand
      ausgerichtet wird. Beide sollen im Kopf platziert werden. Das folgende,
      vollständige Minimalbeispiel liefert genau dies:
\begin{lstcode}
  \documentclass{scrreprt}
  \usepackage{blindtext}
  \usepackage[automark]{scrlayer-scrpage}
  \ihead{\headmark}
  \ohead*{\pagemark}
  \chead{}
  \cfoot[]{}
  \begin{document}
  \blinddocument
  \end{document}
\end{lstcode}
      Das Paket \Package{blindtext}\IndexPackage{blindtext} mit seiner
      Anweisung \Macro{blinddocument}\IndexCmd{blinddocument} wird hier für
      die komfortable Erzeugung eines Beispieldokumentinhalts verwendet.

      Mit den Anweisungen \DescRef{scrlayer-scrpage.cmd.ihead}\IndexCmd{ihead}
      und \DescRef{scrlayer-scrpage.cmd.ohead*}\IndexCmd{ohead*} werden die
      gewünschten Marken platziert. Dabei wird die
      Seitenzahl\iffalse-Marke\fi{} durch die Sternform
      \DescRef{scrlayer-scrpage.cmd.ohead*} %
      \iffalse% Umbruchvarianten
      nicht nur auf den mit
      \DescRef{\LabelBase.pagestyle.scrheadings}\IndexPagestyle{scrheadings}
      gesetzten Seiten, sondern auch für den auf Kapitelanfangsseiten
      automatisch verwendeten \PageStyle{plain}-Seitenstil %
      \else%
      auch für den auf Kapitelanfangsseiten verwendeten Seitenstil %
      \fi%
      \DescRef{\LabelBase.pagestyle.plain.scrheadings}%
      \IndexPagestyle{plain.scrheadings} konfiguriert.%

      Da die Seitenstile bereits mit Marken in der Mitte von Kopf oder
      Fuß vordefiniert sind, werden diese beiden Elemente mit
      \DescRef{scrlayer-scrpage.cmd.chead} und
      \DescRef{scrlayer-scrpage.cmd.cfoot} gelöscht. Hierzu werden leere
      Argumente verwendet. Alternativ dazu hätte man auch
      \DescRef{scrlayer-scrpage-experts.cmd.clearpairofpagestyles}
      \IndexCmd{clearpairofpagestyles}
      \emph{vor} \DescRef{scrlayer-scrpage.cmd.ihead} verwenden können. Diese
      Anweisung wird jedoch erst in
      \autoref{sec:scrlayer-scrpage-experts.pagestyle.pairs} auf
      \DescPageRef{scrlayer-scrpage-experts.cmd.clearpairofpagestyles}
      erklärt werden.
    \end{Example}%

    Bitte beachten Sie, dass das leere optionale Argument bei
    \DescRef{scrlayer-scrpage.cmd.cfoot} im Beispiel nicht gleichbedeutend mit
    dem Weglassen dieses optionalen Arguments ist. Sie sollten das einmal
    selbst ausprobieren und dabei den Fuß der ersten Seite beobachten.%
  \fi

  \IfThisCommonLabelBase{scrlayer-scrpage}{% Umbruchvarianten
    Fortgeschrittene Anwender finden ab
    \IfThisCommonLabelBase{scrlayer-scrpage}{%
      \DescPageRef{scrlayer-scrpage-experts.cmd.righttopmark}}{%
      \DescPageRef{\ThisCommonLabelBase.cmd.righttopmark}} weitere
    Marken-Anweisungen.%
  }{}%
  \iffalse% Umbruchkorrektur
  \ Beispielsweise ist für lexikonartige Dokumente das dort erklärte
  \DescRef{scrlayer-scrpage-experts.cmd.leftfirstmark} und
  \DescRef{scrlayer-scrpage-experts.cmd.rightbotmark} recht nützlich.%
  \fi %
  \EndIndexGroup


  \begin{Declaration}
    \Macro{partmarkformat}
    \Macro{chaptermarkformat}
    \Macro{sectionmarkformat}
    \Macro{subsectionmarkformat}
    \Macro{subsubsectionmarkformat}
    \Macro{paragraphmarkformat}
    \Macro{subparagraphmarkformat}
  \end{Declaration}
  Diese Anweisungen werden von den \KOMAScript-Klassen und auch von
  \IfThisCommonLabelBase{scrlayer}{%
    \Package{scrlayer}%
  }{%
    \hyperref[cha:scrlayer]{\Package{scrlayer}}%
  } intern üblicherweise verwendet, um die Gliederungsnummern der
  automatischen Kolumnentitel zu formatieren. Dabei wird auch der
  \DescRef{maincls.cmd.autodot}-Mechanismus der \KOMAScript-Klassen
  unterstützt. Bei Bedarf können diese Anweisungen umdefiniert werden, um eine
  andere Formatierung der Nummern zu erreichen.%
  \IfThisCommonLabelBase{scrlayer-scrpage}{\iftrue}{%
    \ Siehe dazu gegebenenfalls das Beispiel in
    \autoref{sec:scrlayer-scrpage.pagestyle.content}, auf
    \PageRefxmpl{scrlayer-scrpage.cmd.sectionmarkformat}.%
    \csname iffalse\endcsname%
  }%
    \begin{Example}
      \phantomsection\xmpllabel{cmd.sectionmarkformat}%
      \iffalse
      Wollen Sie auf Abschnittsebene Kolumnentitel ohne Gliederungsnummer,
      geht das beispielsweise so:
      \else
      Angenommen, Sie wollen, dass Abschnittsüberschriften im Kolumnentitel
      ohne Gliederungsnummer gesetzt werden, so ist das ganz
      einfach mit
      \fi
\begin{lstcode}
  \renewcommand*{\sectionmarkformat}{}
\end{lstcode}
      zu erreichen.%
    \end{Example}%
    \ExampleEndFix
  \fi
  % 
  \EndIndexGroup


  \iffree{%
    \begin{Declaration}%
    }{\begin{Declaration}[0]}% Umbruchkorrektur
    \Macro{partmark}\Parameter{Text}
    \Macro{chaptermark}\Parameter{Text}
    \Macro{sectionmark}\Parameter{Text}
    \Macro{subsectionmark}\Parameter{Text}
    \Macro{subsubsectionmark}\Parameter{Text}
    \Macro{paragraphmark}\Parameter{Text}
    \Macro{subparagraphmark}\Parameter{Text}
  \end{Declaration}
  Diese Anweisungen werden intern von den meisten Klassen verwendet, um die
  Marken entsprechend der Gliederungsbefehle zu setzen. Dabei wird als
  Argument lediglich der Text, nicht jedoch die Nummer erwartet. Die Nummer
  wird stattdessen automatisch über den aktuellen Zählerstand ermittelt, falls
  mit nummerierten Überschriften gearbeitet wird.

  Allerdings\textnote{Achtung!} verwenden nicht alle Klassen in allen
  Gliederungsebenen eine solche Anweisung. %
  \IfThisCommonLabelBase{scrlayer-scrpage}{% Umbruchvarianten
    Beispielsweise rufen die Standardklassen\textnote{\KOMAScript{}
      vs. Standardklassen} \Macro{partmark} bei \Macro{part} nicht auf.%
  }{%
    So wird beispielsweise \Macro{partmark} von den
    Standardklassen\textnote{\KOMAScript{} vs. Standardklassen} nie
    aufgerufen, während die \KOMAScript-Klassen selbstverständlich auch
    \Macro{partmark} unterstützen.%
  }

  Falls diese Anweisungen vom Anwender umdefiniert werden, sollte
  er\textnote{Achtung!} unbedingt darauf achten, vor dem Setzen der Nummer
  ebenfalls über \DescRef{maincls.counter.secnumdepth} zu prüfen, ob die
  Nummern auszugeben sind. Dies gilt auch, wenn der Anwender
  \DescRef{maincls.counter.secnumdepth} selbst nicht verändert, weil Pakete
  und Klassen sich eventuell auf die Wirkung von
  \DescRef{maincls.counter.secnumdepth} verlassen!

  Das Paket \IfThisCommonLabelBase{scrlayer}{%
    \Package{scrlayer}%
  }{%
    \hyperref[cha:scrlayer]{\Package{scrlayer}}%
  } definiert diese Anweisungen außerdem bei jedem Aufruf von
  \DescRef{scrlayer.cmd.automark} oder \DescRef{scrlayer.cmd.manualmark} oder
  den entsprechenden Optionen teilweise neu, um so die gewünschten
  automatischen oder manuellen Kolumnentitel zu erreichen.%
  \EndIndexGroup


  \begin{Declaration}
    \Macro{markleft}\Parameter{linke Marke}
    \Macro{markright}\Parameter{rechte Marke}
    \Macro{markboth}\Parameter{linke Marke}\Parameter{rechte Marke}
    \Macro{markdouble}\Parameter{Marke}
  \end{Declaration}
  Unabhängig davon, ob gerade mit manuellen oder automatischen Kolumnentiteln
  gearbeitet wird, kann man jederzeit die \PName{linke Marke} oder
  \PName{rechte Marke} mit einer dieser Anweisungen setzen. Dabei ist zu
  beachten, dass die resultierende linke Marke in
  \Macro{leftmark}\IndexCmd{leftmark}\important{\Macro{leftmark}} die letzte
  auf der entsprechenden Seite gesetzte Marke ist, während die resultierende
  rechte Marke in
  \Macro{rightmark}\IndexCmd{rightmark}\important{\Macro{rightmark}} die erste
  auf der entsprechenden Seite gesetzte Marke ausgibt. Näheres dazu ist den
  weiterführenden Erklärungen zu \DescRef{maincls-experts.cmd.rightmark} in
  \autoref{sec:maincls-experts.addInfos},
  \DescPageRef{maincls-experts.cmd.rightmark} oder zu
  \DescRef{scrlayer.cmd.rightfirstmark}\IfThisCommonLabelBase{scrlayer}{}{ in
    \autoref{sec:scrlayer.pagestyle.content}},
  \DescPageRef{scrlayer.cmd.rightfirstmark} zu entnehmen.

  Wird mit manuellen Kolumnentiteln\Index{Kolumnentitel>manuell} gearbeitet,
  so bleiben die Marken gültig, bis sie durch erneute Verwendung der
  entsprechenden Anweisung explizit ersetzt werden. Bei automatischen
  Kolumnentiteln können Marken hingegen je nach Konfigurierung des
  Automatismus ihre Gültigkeit mit einer der nächsten Gliederungsüberschriften
  verlieren.

  Auch im Zusammenhang mit den Sternvarianten der Gliederungsbefehle können
  diese Anweisungen nützlich sein.%
  \IfThisCommonLabelBase{scrlayer-scrpage}{\iftrue}{%
    \ Ausführliche Beispiele für die Verwendung von \Macro{markboth} mit dem
    von \IfThisCommonLabelBase{scrlayer-scrpage}{%
      \hyperref[cha:scrlayer]{\Package{scrlayer}}%
    }{%
      \Package{scrlayer}%
    } abgeleiteten Paket \IfThisCommonLabelBase{scrlayer-scrpage}{%
      \Package{scrlayer-scrpage}%
    }{%
      \hyperref[cha:scrlayer-scrpage]{\Package{scrlayer-scrpage}}%
    } sind in \autoref{sec:scrlayer-scrpage.pagestyle.content}, ab
    \PageRefxmpl{scrlayer-scrpage.cmd.markboth} zu finden.%
    \csname iffalse\endcsname%
  }%
    \begin{Example}
      \phantomsection\xmpllabel{cmd.markboth}%
      Angenommen, Sie schreiben noch vor dem Inhaltsverzeichnis ein Vorwort
      über mehrere Seiten, das jedoch im Inhaltsverzeichnis nicht auf"|tauchen
      soll. Da Sie aber Trennlinien im Kopf verwenden, soll der Kolumnentitel
      das Vorwort dennoch zeigen:
\begin{lstcode}
  \documentclass[headsepline]{book}
  \usepackage[automark]{scrlayer-scrpage}
  \usepackage{blindtext}
  \begin{document}
  \chapter*{Vorwort}
  \markboth{Vorwort}{Vorwort}
  \blindtext[20]
  \tableofcontents
  \blinddocument
  \end{document}
\end{lstcode}
      Zunächst erscheint das Ergebnis wunschgemäß. Vielleicht erst beim
      zweiten Blick fällt aber auf, dass der Kolumnentitel »\texttt{Vorwort}«
      im Gegensatz zu den übrigen Kolumnentiteln nicht im Versalsatz
      erscheint. Das ist jedoch leicht zu ändern:
\begin{lstcode}
  \documentclass[headsepline]{book}
  \usepackage[automark]{scrlayer-scrpage}
  \usepackage{blindtext}
  \begin{document}
  \chapter*{Vorwort}
  \markboth{\MakeMarkcase{Vorwort}}%
           {\MakeMarkcase{Vorwort}}
  \blindtext[20]
  \tableofcontents
  \blinddocument
  \end{document}
\end{lstcode}
      Wie zu sehen ist, wurde \DescRef{scrlayer.cmd.MakeMarkcase}%
      \IndexCmd{MakeMarkcase}\important{\DescRef{scrlayer.cmd.MakeMarkcase}}
      verwendet, um auch den manuell korrigierten Kolumnentitel des Vorworts
      entsprechend der automatischen Kolumnentitel des restlichen Dokuments
      anzupassen.

      Verschieben Sie nun einmal \DescRef{maincls.cmd.tableofcontents}%
      \Index{Inhaltsverzeichnis}\IndexCmd{tableofcontents}%
      \important{\DescRef{maincls.cmd.tableofcontents}} vor das Vorwort und
      entfernen Sie die \Macro{markboth}-Anweisung. Sie werden entdecken, dass
      das Vorwort als Kolumnentitel nun »\texttt{CONTENTS}« trägt. Das liegt
      an einer Eigenart von \DescRef{maincls.cmd.chapter*}%
      \IndexCmd{chapter*}\important{\DescRef{maincls.cmd.chapter*}} (siehe
      auch in \autoref{sec:maincls.structure} auf
      \DescPageRef{maincls.cmd.chapter*}). Soll hier stattdessen kein
      Kolumnentitel erscheinen, so ist dies sehr einfach mit \Macro{markboth}
      mit zwei leeren Argumenten zu erreichen:
\begin{lstcode}
  \documentclass[headsepline]{book}
  \usepackage[automark]{scrlayer-scrpage}
  \usepackage{blindtext}
  \begin{document}
  \tableofcontents
  \chapter*{Vorwort}
  \markboth{}{}
  \blindtext[20]
  \blinddocument
  \end{document}
\end{lstcode}
    \end{Example}
  \fi%
  Die\ChangedAt{v3.28}{\Package{scrlayer}} Anweisung
  \Macro{markdouble}\important{\Macro{markdouble}} setzt sowohl die linke als
  auch rechte Marke auf denselben Inhalt. Damit ist
  \Macro{markdouble}\Parameter{Marke} eine abkürzende Schreibweise für
  \Macro{markboth}\Parameter{Marke}\Parameter{Marke} mit zwei identischen
  Argumenten.%
  \EndIndexGroup
\fi


\IfThisCommonLabelBase{scrlayer-scrpage}{\iffalse}{\csname iftrue\endcsname}
  \begin{Declaration}
    \Macro{GenericMarkFormat}\Parameter{Gliederungsname}
  \end{Declaration}
  Diese Anweisung wird in der Voreinstellung zur Formatierung aller
  Gliederungsnummern in automatischen Kolumnentiteln unterhalb der
  Unterabschnitte und bei Klassen ohne \DescRef{maincls.cmd.chapter}
  zusätzlich auch für die Ebene der Abschnitte und Unterabschnitte verwendet,
  soweit die entsprechenden Mark-Anweisungen nicht bereits anderweitig
  definiert sind. Dabei verwendet die Anweisung in der Voreinstellung
  \Macro{@seccntmarkformat}\IndexCmd{@seccntmarkformat}%
  \important{\Macro{@seccntmarkformat}}, wenn eine solche interne Anweisung
  wie bei den \KOMAScript-Klassen definiert ist. Anderenfalls wird mit
  \Macro{@seccntformat}\IndexCmd{@seccntformat}\important{\Macro{@seccntformat}}
  eine Anweisung verwendet, die bereits vom \LaTeX-Kern für Klassen und Pakete
  bereitgestellt und von \KOMAScript{} etwas modifiziert wird. Als Argument
  erwartet \Macro{GenericMarkFormat} den Namen der Gliederung, also
  beispielsweise \PValue{chapter} oder \PValue{section} \emph{ohne}
  vorangestellten umgekehrten Schrägstrich (engl. \emph{backslash}).

  Durch Umdefinierung dieser Anweisung kann damit die Standardformatierung
  aller Gliederungsnummern im Kolumnentitel geändert werden, die darauf
  zurückgreifen. Ebenso kann eine Klasse darüber eine andere
  Standardformatierung vorgeben, ohne alle Befehle einzeln ändern zu müssen.%
  \IfThisCommonLabelBase{scrlayer-scrpage-experts}{\iftrue}{%
    \par %
    Ein ausführliches Beispiel für das Zusammenspiel der Anweisung
    \Macro{GenericMarkFormat} mit den auf
    \DescPageRef{\ThisCommonLabelBase.cmd.chaptermark} erklärten Anweisungen
    \DescRef{\ThisCommonLabelBase.cmd.sectionmarkformat} und
    \DescRef{\ThisCommonLabelBase.cmd.subsectionmarkformat} beziehungsweise
    \DescRef{\ThisCommonLabelBase.cmd.chaptermarkformat} bei Verwendung des
    von \IfThisCommonLabelBase{scrlayer}{%
      \Package{scrlayer}%
    }{%
      \hyperref[cha:scrlayer]{\Package{scrlayer}}%
    } abgeleiteten Pakets \IfThisCommonLabelBase{scrlayer-scrpage}{%
      \Package{scrlayer-scrpage}%
    }{%
      \hyperref[cha:scrlayer-scrpage]{\Package{scrlayer-scrpage}}%
    } ist in \autoref{sec:scrlayer-scrpage-experts.pagestyle.content}, ab
    \PageRefxmpl{scrlayer-scrpage-experts.cmd.GenericMarkFormat} zu finden.%
    \csname iffalse\endcsname}%
    \begin{Example}
      \phantomsection
      \xmpllabel{cmd.GenericMarkFormat}%
      Angenommen, Sie wollen, dass bei allen Gliederungsnummern im
      Kolumnentitel eines Artikels die Nummer als weiße Schrift auf einem
      schwarzen Kasten ausgegeben wird. Da bei Artikeln mit Klasse
      \Class{article} die Anweisungen \DescRef{scrlayer.cmd.sectionmarkformat}%
      \IndexCmd{sectionmarkformat}%
      \important{%
        \DescRef{scrlayer.cmd.sectionmarkformat}\\
        \DescRef{scrlayer.cmd.subsectionmarkformat}} und
      \DescRef{scrlayer.cmd.subsectionmarkformat}%
      \IndexCmd{subsectionmarkformat} von \IfThisCommonLabelBase{scrlayer}{%
    \Package{scrlayer}%
  }{%
    \hyperref[cha:scrlayer]{\Package{scrlayer}}%
  } mit Hilfe von
      \Macro{GenericMarkFormat} definiert werden, genügt dafür die
      entsprechende Umdefinierung dieser einen Anweisung:
\begin{lstcode}[moretexcs={colorbox,textcolor}]
  \documentclass{article}
  \usepackage{blindtext}
  \usepackage[automark]{scrlayer-scrpage}
  \usepackage{xcolor}
  \newcommand*{\numberbox}[1]{%
    \colorbox{black}{\strut~\textcolor{white}{#1}~}%
  }
  \renewcommand*{\GenericMarkFormat}[1]{%
    \protect\numberbox{\csname the#1\endcsname}%
    \enskip
  }
  \begin{document}
  \blinddocument
  \end{document}
\end{lstcode}
      % Umbruchkorrektur (mit Ausgleich zur linken Seite):
      \iffree{}{\enlargethispage{-\baselineskip}\pagebreak}%
      Für die Farbumschaltungen werden Anweisungen des Pakets
      \Package{xcolor}\IndexPackage{xcolor} verwendet. Näheres dazu ist der
      Anleitung zum Paket zu entnehmen (siehe \cite{package:xcolor}). 
      \iffalse \par\fi % Umbruchkorrektur (zusammen mit nachfolgender)
      Außerdem wird eine unsichtbare Stütze mit \Macro{strut}
      eingefügt.%
      \iffalse Diese Anweisung sollte in keiner ausführlichen
      \LaTeX-Einführung fehlen. \fi % siehe oben

      Für den Kasten mit der Nummer wird eine eigene Hilfsanweisung
      \Macro{numberbox} definiert. Diese wird in der Umdefinierung von
      \Macro{GenericMarkFormat} mit
      \Macro{protect}\IndexCmd{protect}\important{\Macro{protect}} vor der
      Expansion geschützt. Dies ist notwendig, weil sonst durch das
      \Macro{MakeUppercase}\IndexCmd{MakeUppercase}%
      \important{\Macro{MakeUppercase}} für den Versalsatz der Kolumnentitel
      nicht mehr die Farben »\texttt{black}« und »\texttt{white}«, sondern die
      Farben »\texttt{BLACK}« und »\texttt{WHITE}« verlangt würden, die gar
      nicht definiert sind. Alternativ könnte man \Macro{numberbox} auch mit
      Hilfe von \Macro{DeclareRobustCommand*} statt mit \Macro{newcommand*}
      definieren (siehe \cite{latex:clsguide}).

      Wollte man dasselbe mit einer \KOMAScript-Klasse oder mit den
      Standardklassen \Class{book} oder \Class{report} erreichen, so müsste
      man übrigens zusätzlich
      \DescRef{scrlayer.cmd.sectionmarkformat}%
      \IndexCmd{sectionmarkformat}%
      \important{\DescRef{scrlayer.cmd.sectionmarkformat}} und --
      je nach Klasse --
      \DescRef{scrlayer.cmd.subsectionmarkformat}%
      \IndexCmd{subsectionmarkformat}%
      \important{\DescRef{scrlayer.cmd.subsectionmarkformat}}
      beziehungsweise \DescRef{scrlayer.cmd.chaptermarkformat}%
      \IndexCmd{chaptermarkformat}%
      \important{\DescRef{scrlayer.cmd.chaptermarkformat}}
      umdefinieren, da diese bei Verwendung der genannten Klassen
      \Macro{GenericMarkFormat} nicht verwenden:
\begin{lstcode}[moretexcs={colorbox,textcolor}]
  \documentclass[headheight=19.6pt]{scrbook}
  \usepackage{blindtext}
  \usepackage[automark]{scrlayer-scrpage}
  \usepackage{xcolor}
  \newcommand*{\numberbox}[1]{%
    \colorbox{black}{\strut~\textcolor{white}{#1}~}%
  }
  \renewcommand*{\GenericMarkFormat}[1]{%
    \protect\numberbox{\csname the#1\endcsname}%
    \enskip
  }
  \renewcommand*{\chaptermarkformat}{%
    \GenericMarkFormat{chapter}%
  }
  \renewcommand*{\sectionmarkformat}{%
    \GenericMarkFormat{section}%
  }
  \begin{document}
  \blinddocument
  \end{document}
\end{lstcode}
    Über Option \DescRef{typearea.option.headheight} wird dabei auch die
    Warnung beseitigt, die im vorherigen Beispiel noch erzeugt wurde.%
    \end{Example}
  \fi%
  \EndIndexGroup
\fi
  

\IfThisCommonLabelBase{scrlayer-scrpage}{\iffalse}{\csname iftrue\endcsname}
  \begin{Declaration}
    \Macro{@mkleft}\Parameter{linke Marke}%
    \Macro{@mkright}\Parameter{rechte Marke}%
    \Macro{@mkdouble}\Parameter{Marke}%
    \Macro{@mkboth}\Parameter{linke Marke}\Parameter{rechte Marke}%
  \end{Declaration}
  Innerhalb der Klassen und Pakete kommt es vor, dass Kolumnentitel nur dann
  gesetzt werden sollen, wenn automatische Kolumnentitel (siehe Option
  \DescRef{scrlayer.option.automark} und Anweisung
  \DescRef{scrlayer.cmd.automark} auf
  \DescPageRef{scrlayer.cmd.automark}) aktiviert sind. Bei
  den Standardklassen geht dies ausschließlich über \Macro{@mkboth}. Diese
  Anweisung entspricht entweder \Macro{@gobbletwo}, einer Anweisung, die ihre
  beiden Argumente vernichtet, oder \DescRef{scrlayer.cmd.markboth},
  einer Anweisung, mit der sowohl eine \PName{linke Marke} als auch eine
  \PName{rechte Marke} gesetzt wird. Pakete wie \Package{babel} hängen sich
  ebenfalls in \Macro{@mkboth} ein, um beispielsweise noch eine
  Sprachumschaltung im Kolumnentitel vorzunehmen.

  Will man nun jedoch nur eine \PName{linke Marke} oder nur eine \PName{rechte
    Marke} setzen, ohne die jeweils andere Marke zu verändern, so fehlen
  entsprechende Anweisungen. Das Paket \IfThisCommonLabelBase{scrlayer}{%
    \Package{scrlayer}%
  }{%
    \hyperref[cha:scrlayer]{\Package{scrlayer}}%
  } selbst benötigt entsprechende Anweisungen beispielsweise im Rahmen der
  automatischen Kolumnentitel. Sind \Macro{@mkleft} zum Setzen nur der
  \PName{linken Marke}, \Macro{@mkright} zum Setzen nur der \PName{rechten
    Marke} oder \Macro{@mkdouble} zum Setzen sowohl der rechten als auch der
  linken \PName{Marke} mit demselben Inhalt beim Laden von
  \IfThisCommonLabelBase{scrlayer}{%
    \Package{scrlayer}%
  }{%
    \hyperref[cha:scrlayer]{\Package{scrlayer}}%
  } nicht definiert, so werden sie vom Paket selbst definiert. Dabei wird eine
  Definition gewählt, die am Zustand von \Macro{@mkboth} erkennt, ob mit
  automatischen Kolumnentiteln gearbeitet wird. Nur in diesem Fall setzen die
  Befehle auch eine entsprechende Marke.%

  Klassen- und Paketautoren können ebenfalls auf die passende der vier
  Anweisungen zurückgreifen, wenn sie linke oder rechte Marken setzen und
  dies auf den Fall beschränken wollen, dass mit automatischen Kolumnentiteln
  gearbeitet wird.%
  \EndIndexGroup%
\fi

\IfThisCommonLabelBase{scrlayer}{%
  Zu weiteren Möglichkeiten zur Beeinflussung der Inhalte von Seitenstilen
  siehe auch \autoref{sec:scrlayer-scrpage.pagestyle.content},
  \autopageref{sec:scrlayer-scrpage.pagestyle.content}.%
}{}%
\EndIndexGroup

%%% Local Variables: 
%%% mode: latex
%%% TeX-master: "scrguide-de.tex"
%%% coding: utf-8
%%% ispell-local-dictionary: "de_DE"
%%% eval: (flyspell-mode 1)
%%% End: 

%  LocalWords:  Seitenstilen Ebenenmodell scrpage headings myheadings plain
%  LocalWords:  empty scrlayer Seitenstil Rückgriffe Gliederungsnummern
%  LocalWords:  Standardklassen Seitenstile konsistenteren Befehlssatzes
%  LocalWords:  Einstellmöglichkeiten Seiteninhalts Gliederungsüberschrift
%  LocalWords:  Maßlinien Entwurfsphase Unterabschnittsüberschrift Versalsatz
%  LocalWords:  Abschnittsüberschrift Kapitelüberschrift Glyphenanalyse
%  LocalWords:  Hilfsanweisung content Expertenwissen Gliederungsebenen
%  LocalWords:  Gliederungsüberschriften Kapitelüberschriften Kolumnentitel
%  LocalWords:  Abschnittsüberschriften Gliederungsnummer Kapitelanfangsseite
%  LocalWords:  Kolumnentiteln Abschnittsebene Paketautoren
